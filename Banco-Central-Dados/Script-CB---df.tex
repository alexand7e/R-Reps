% Options for packages loaded elsewhere
\PassOptionsToPackage{unicode}{hyperref}
\PassOptionsToPackage{hyphens}{url}
%
\documentclass[
]{article}
\usepackage{amsmath,amssymb}
\usepackage{lmodern}
\usepackage{iftex}
\ifPDFTeX
  \usepackage[T1]{fontenc}
  \usepackage[utf8]{inputenc}
  \usepackage{textcomp} % provide euro and other symbols
\else % if luatex or xetex
  \usepackage{unicode-math}
  \defaultfontfeatures{Scale=MatchLowercase}
  \defaultfontfeatures[\rmfamily]{Ligatures=TeX,Scale=1}
\fi
% Use upquote if available, for straight quotes in verbatim environments
\IfFileExists{upquote.sty}{\usepackage{upquote}}{}
\IfFileExists{microtype.sty}{% use microtype if available
  \usepackage[]{microtype}
  \UseMicrotypeSet[protrusion]{basicmath} % disable protrusion for tt fonts
}{}
\makeatletter
\@ifundefined{KOMAClassName}{% if non-KOMA class
  \IfFileExists{parskip.sty}{%
    \usepackage{parskip}
  }{% else
    \setlength{\parindent}{0pt}
    \setlength{\parskip}{6pt plus 2pt minus 1pt}}
}{% if KOMA class
  \KOMAoptions{parskip=half}}
\makeatother
\usepackage{xcolor}
\usepackage[margin=1in]{geometry}
\usepackage{color}
\usepackage{fancyvrb}
\newcommand{\VerbBar}{|}
\newcommand{\VERB}{\Verb[commandchars=\\\{\}]}
\DefineVerbatimEnvironment{Highlighting}{Verbatim}{commandchars=\\\{\}}
% Add ',fontsize=\small' for more characters per line
\usepackage{framed}
\definecolor{shadecolor}{RGB}{248,248,248}
\newenvironment{Shaded}{\begin{snugshade}}{\end{snugshade}}
\newcommand{\AlertTok}[1]{\textcolor[rgb]{0.94,0.16,0.16}{#1}}
\newcommand{\AnnotationTok}[1]{\textcolor[rgb]{0.56,0.35,0.01}{\textbf{\textit{#1}}}}
\newcommand{\AttributeTok}[1]{\textcolor[rgb]{0.77,0.63,0.00}{#1}}
\newcommand{\BaseNTok}[1]{\textcolor[rgb]{0.00,0.00,0.81}{#1}}
\newcommand{\BuiltInTok}[1]{#1}
\newcommand{\CharTok}[1]{\textcolor[rgb]{0.31,0.60,0.02}{#1}}
\newcommand{\CommentTok}[1]{\textcolor[rgb]{0.56,0.35,0.01}{\textit{#1}}}
\newcommand{\CommentVarTok}[1]{\textcolor[rgb]{0.56,0.35,0.01}{\textbf{\textit{#1}}}}
\newcommand{\ConstantTok}[1]{\textcolor[rgb]{0.00,0.00,0.00}{#1}}
\newcommand{\ControlFlowTok}[1]{\textcolor[rgb]{0.13,0.29,0.53}{\textbf{#1}}}
\newcommand{\DataTypeTok}[1]{\textcolor[rgb]{0.13,0.29,0.53}{#1}}
\newcommand{\DecValTok}[1]{\textcolor[rgb]{0.00,0.00,0.81}{#1}}
\newcommand{\DocumentationTok}[1]{\textcolor[rgb]{0.56,0.35,0.01}{\textbf{\textit{#1}}}}
\newcommand{\ErrorTok}[1]{\textcolor[rgb]{0.64,0.00,0.00}{\textbf{#1}}}
\newcommand{\ExtensionTok}[1]{#1}
\newcommand{\FloatTok}[1]{\textcolor[rgb]{0.00,0.00,0.81}{#1}}
\newcommand{\FunctionTok}[1]{\textcolor[rgb]{0.00,0.00,0.00}{#1}}
\newcommand{\ImportTok}[1]{#1}
\newcommand{\InformationTok}[1]{\textcolor[rgb]{0.56,0.35,0.01}{\textbf{\textit{#1}}}}
\newcommand{\KeywordTok}[1]{\textcolor[rgb]{0.13,0.29,0.53}{\textbf{#1}}}
\newcommand{\NormalTok}[1]{#1}
\newcommand{\OperatorTok}[1]{\textcolor[rgb]{0.81,0.36,0.00}{\textbf{#1}}}
\newcommand{\OtherTok}[1]{\textcolor[rgb]{0.56,0.35,0.01}{#1}}
\newcommand{\PreprocessorTok}[1]{\textcolor[rgb]{0.56,0.35,0.01}{\textit{#1}}}
\newcommand{\RegionMarkerTok}[1]{#1}
\newcommand{\SpecialCharTok}[1]{\textcolor[rgb]{0.00,0.00,0.00}{#1}}
\newcommand{\SpecialStringTok}[1]{\textcolor[rgb]{0.31,0.60,0.02}{#1}}
\newcommand{\StringTok}[1]{\textcolor[rgb]{0.31,0.60,0.02}{#1}}
\newcommand{\VariableTok}[1]{\textcolor[rgb]{0.00,0.00,0.00}{#1}}
\newcommand{\VerbatimStringTok}[1]{\textcolor[rgb]{0.31,0.60,0.02}{#1}}
\newcommand{\WarningTok}[1]{\textcolor[rgb]{0.56,0.35,0.01}{\textbf{\textit{#1}}}}
\usepackage{longtable,booktabs,array}
\usepackage{calc} % for calculating minipage widths
% Correct order of tables after \paragraph or \subparagraph
\usepackage{etoolbox}
\makeatletter
\patchcmd\longtable{\par}{\if@noskipsec\mbox{}\fi\par}{}{}
\makeatother
% Allow footnotes in longtable head/foot
\IfFileExists{footnotehyper.sty}{\usepackage{footnotehyper}}{\usepackage{footnote}}
\makesavenoteenv{longtable}
\usepackage{graphicx}
\makeatletter
\def\maxwidth{\ifdim\Gin@nat@width>\linewidth\linewidth\else\Gin@nat@width\fi}
\def\maxheight{\ifdim\Gin@nat@height>\textheight\textheight\else\Gin@nat@height\fi}
\makeatother
% Scale images if necessary, so that they will not overflow the page
% margins by default, and it is still possible to overwrite the defaults
% using explicit options in \includegraphics[width, height, ...]{}
\setkeys{Gin}{width=\maxwidth,height=\maxheight,keepaspectratio}
% Set default figure placement to htbp
\makeatletter
\def\fps@figure{htbp}
\makeatother
\setlength{\emergencystretch}{3em} % prevent overfull lines
\providecommand{\tightlist}{%
  \setlength{\itemsep}{0pt}\setlength{\parskip}{0pt}}
\setcounter{secnumdepth}{-\maxdimen} % remove section numbering
\usepackage{booktabs}
\usepackage{longtable}
\usepackage{array}
\usepackage{multirow}
\usepackage{wrapfig}
\usepackage{float}
\usepackage{colortbl}
\usepackage{pdflscape}
\usepackage{tabu}
\usepackage{threeparttable}
\usepackage{threeparttablex}
\usepackage[normalem]{ulem}
\usepackage{makecell}
\usepackage{xcolor}
\ifLuaTeX
  \usepackage{selnolig}  % disable illegal ligatures
\fi
\IfFileExists{bookmark.sty}{\usepackage{bookmark}}{\usepackage{hyperref}}
\IfFileExists{xurl.sty}{\usepackage{xurl}}{} % add URL line breaks if available
\urlstyle{same} % disable monospaced font for URLs
\hypersetup{
  pdftitle={Aplicação do Índice de Hirschman-Herfindahl no setor financeiro brasileiro},
  pdfauthor={Alexandre Barros},
  hidelinks,
  pdfcreator={LaTeX via pandoc}}

\title{Aplicação do Índice de Hirschman-Herfindahl no setor financeiro
brasileiro}
\usepackage{etoolbox}
\makeatletter
\providecommand{\subtitle}[1]{% add subtitle to \maketitle
  \apptocmd{\@title}{\par {\large #1 \par}}{}{}
}
\makeatother
\subtitle{Estudo para os anos de 2017 a 2021}
\author{Alexandre Barros}
\date{30 de agosto de 2022}

\begin{document}
\maketitle

\hypertarget{introduuxe7uxe3o}{%
\section{Introdução}\label{introduuxe7uxe3o}}

O objetivo deste documento é apresentar, metodicamente e em etapas
explicadas no R, a utilização de dados do setor financeiro brasileiro,
dispobibilizados pelo Banco Central do Brasil, nos {[}relatórios do
IF.Período{]}{]}(\url{https://www3.bcb.gov.br/ifPeríodo/\#}).

Para tanto, além desta breve introdução, o documento se desdobrará,
sempre explicando os comandos, no método e nos resultados aplicados com
dados dos primeiros trimestres de 2017 a 2021, a fim de obter
informações do grau de concentração neste setor.

\hypertarget{base-de-dados}{%
\section{Base de Dados}\label{base-de-dados}}

Os dados do portal IF.Período, do BC, disponibilizam informações
diversas, resumidas e desdobradas, de diversas instituições (cerca de
1500) e em diversos segmentos do setor financeiro. O primeiro passo
corresponde a importação desses dados (obtidos em .csv) de maneira
off-line, após-realizados o download.

No R, utilizamos as funções do pacote Dplyr de importação. Abaixo,
usa-se a função \emph{read.csv()} para importar do diretório os
arquivos.

\begin{Shaded}
\begin{Highlighting}[]
\CommentTok{\#2017 \# }
\NormalTok{dados\_17 }\OtherTok{\textless{}{-}} \FunctionTok{read.csv}\NormalTok{(}\StringTok{"dados bancários 2017.csv"}\NormalTok{, }\AttributeTok{sep =} \StringTok{";"}\NormalTok{, }\AttributeTok{skip =} \DecValTok{1}\NormalTok{) }\SpecialCharTok{\%\textgreater{}\%} 
  \FunctionTok{select}\NormalTok{(}\DecValTok{1}\NormalTok{,}\DecValTok{6}\NormalTok{,}\DecValTok{17}\NormalTok{,}\DecValTok{28}\NormalTok{) }\SpecialCharTok{\%\textgreater{}\%} \FunctionTok{filter}\NormalTok{(X}\FloatTok{.5} \SpecialCharTok{!=} \StringTok{""}\NormalTok{) }\SpecialCharTok{\%\textgreater{}\%} \FunctionTok{mutate}\NormalTok{(Período }\OtherTok{=} \StringTok{"03/2017"}\NormalTok{) }
\CommentTok{\#2018}
\NormalTok{dados\_18 }\OtherTok{\textless{}{-}} \FunctionTok{read.csv}\NormalTok{(}\StringTok{"dados bancários 2018.csv"}\NormalTok{, }\AttributeTok{sep =} \StringTok{";"}\NormalTok{, }\AttributeTok{skip =} \DecValTok{1}\NormalTok{) }\SpecialCharTok{\%\textgreater{}\%} 
  \FunctionTok{select}\NormalTok{(}\DecValTok{1}\NormalTok{,}\DecValTok{6}\NormalTok{,}\DecValTok{17}\NormalTok{,}\DecValTok{28}\NormalTok{) }\SpecialCharTok{\%\textgreater{}\%} \FunctionTok{filter}\NormalTok{(X}\FloatTok{.5} \SpecialCharTok{!=} \StringTok{""}\NormalTok{) }\SpecialCharTok{\%\textgreater{}\%} \FunctionTok{mutate}\NormalTok{(Período }\OtherTok{=} \StringTok{"03/2018"}\NormalTok{)}
\CommentTok{\#2019}
\NormalTok{dados\_19 }\OtherTok{\textless{}{-}} \FunctionTok{read.csv}\NormalTok{(}\StringTok{"dados bancários 2019.csv"}\NormalTok{, }\AttributeTok{sep =} \StringTok{";"}\NormalTok{, }\AttributeTok{skip =} \DecValTok{1}\NormalTok{) }\SpecialCharTok{\%\textgreater{}\%} 
  \FunctionTok{select}\NormalTok{(}\DecValTok{1}\NormalTok{,}\DecValTok{6}\NormalTok{,}\DecValTok{17}\NormalTok{,}\DecValTok{28}\NormalTok{) }\SpecialCharTok{\%\textgreater{}\%} \FunctionTok{filter}\NormalTok{(X}\FloatTok{.5} \SpecialCharTok{!=} \StringTok{""}\NormalTok{) }\SpecialCharTok{\%\textgreater{}\%} \FunctionTok{mutate}\NormalTok{(Período }\OtherTok{=} \StringTok{"03/2019"}\NormalTok{)}
\CommentTok{\#2020}
\NormalTok{dados\_20 }\OtherTok{\textless{}{-}} \FunctionTok{read.csv}\NormalTok{(}\StringTok{"dados bancários 2020.csv"}\NormalTok{, }\AttributeTok{sep =} \StringTok{";"}\NormalTok{, }\AttributeTok{skip =} \DecValTok{1}\NormalTok{) }\SpecialCharTok{\%\textgreater{}\%} 
  \FunctionTok{select}\NormalTok{(}\DecValTok{1}\NormalTok{,}\DecValTok{6}\NormalTok{,}\DecValTok{17}\NormalTok{,}\DecValTok{28}\NormalTok{) }\SpecialCharTok{\%\textgreater{}\%} \FunctionTok{filter}\NormalTok{(X}\FloatTok{.5} \SpecialCharTok{!=} \StringTok{""}\NormalTok{) }\SpecialCharTok{\%\textgreater{}\%} \FunctionTok{mutate}\NormalTok{(Período }\OtherTok{=} \StringTok{"03/2020"}\NormalTok{)}
\CommentTok{\#2021}
\NormalTok{dados\_21 }\OtherTok{\textless{}{-}} \FunctionTok{read.csv}\NormalTok{(}\StringTok{"dados bancários 2021.csv"}\NormalTok{, }\AttributeTok{sep =} \StringTok{";"}\NormalTok{, }\AttributeTok{skip =} \DecValTok{1}\NormalTok{) }\SpecialCharTok{\%\textgreater{}\%} 
  \FunctionTok{select}\NormalTok{(}\DecValTok{1}\NormalTok{,}\DecValTok{6}\NormalTok{,}\DecValTok{17}\NormalTok{,}\DecValTok{28}\NormalTok{) }\SpecialCharTok{\%\textgreater{}\%} \FunctionTok{filter}\NormalTok{(X}\FloatTok{.5} \SpecialCharTok{!=} \StringTok{""}\NormalTok{) }\SpecialCharTok{\%\textgreater{}\%} \FunctionTok{mutate}\NormalTok{(Período }\OtherTok{=} \StringTok{"03/2021"}\NormalTok{)}
\end{Highlighting}
\end{Shaded}

Além da importação, selecionou-se apenas as colunas ou variáveis: 1)
nome da instituição, 2) subsetor da instiuição, 3) operações de crédito
líquidas, e 4) Ativo total das insituições; isto para cada período.

Os comandos abaixos organizam os dados em uma base consolidada para sua
utilização.

1 - Consolidando todos os dados em um só data.frame.

\begin{Shaded}
\begin{Highlighting}[]
\NormalTok{Dados.com }\OtherTok{\textless{}{-}} \FunctionTok{bind\_rows}\NormalTok{(dados\_17, dados\_18, dados\_19, dados\_20, dados\_21)}
\end{Highlighting}
\end{Shaded}

2 - Alterando o nome das variáveis.

\begin{Shaded}
\begin{Highlighting}[]
\FunctionTok{names}\NormalTok{(Dados.com) }\OtherTok{\textless{}{-}} \FunctionTok{c}\NormalTok{(}\StringTok{"Instituições"}\NormalTok{, }
                      \StringTok{"Tipo"}\NormalTok{, }
                      \StringTok{"Operações de Crédito"}\NormalTok{,}
                      \StringTok{"Ativo total"}\NormalTok{, }
                      \StringTok{"Período"}\NormalTok{)}
\end{Highlighting}
\end{Shaded}

3 - Retirando demais objetos do R (retirando excessos).

\begin{Shaded}
\begin{Highlighting}[]
\NormalTok{gdata}\SpecialCharTok{::}\FunctionTok{keep}\NormalTok{(Dados.com, }\AttributeTok{sure =}\NormalTok{ T)}
\end{Highlighting}
\end{Shaded}

\hypertarget{analisando-os-dados}{%
\subsection{Analisando os dados}\label{analisando-os-dados}}

Agora que consolidamos os dados, para os respectivos períodos, no objeto
a que chamamos \emph{Dados.com}, podemos analisar preliminarmente estes
dados.

O próximo passo é será revisar os dados. Ou seja, ocorre que no processo
de importação, o R nem sempre reconhece o tipo correto de dados. É dever
nosso completar esses dados. Abaixo o script responsável por aprimorar
as variáveis existentes, usando a função \emph{mutate()}, que cria novas
variáveis a partir de variáveis existentes.

Como se pode ver, adicionamos os Subsetores de cada setor de maneira a
melhor analisar os dados, com o auxílio da função case\_when, que usa de
operações lógicas para identificar corretamente o subsetor da
instituição.

\begin{Shaded}
\begin{Highlighting}[]
\NormalTok{Dados.com }\SpecialCharTok{\%\textless{}\textgreater{}\%} \FunctionTok{mutate}\NormalTok{( }\CommentTok{\# }
  \StringTok{\textasciigrave{}}\AttributeTok{Ativo total}\StringTok{\textasciigrave{}} \OtherTok{=} 
    \FunctionTok{as.numeric}\NormalTok{(}\FunctionTok{str\_replace\_all}\NormalTok{(}\StringTok{\textasciigrave{}}\AttributeTok{Ativo total}\StringTok{\textasciigrave{}}\NormalTok{,}\StringTok{"}\SpecialCharTok{\textbackslash{}\textbackslash{}}\StringTok{."}\NormalTok{, }\StringTok{""}\NormalTok{))}\SpecialCharTok{/}\DecValTok{1000}\NormalTok{,}
  
  \StringTok{\textasciigrave{}}\AttributeTok{Operações de Crédito}\StringTok{\textasciigrave{}} \OtherTok{=} 
    \FunctionTok{as.numeric}\NormalTok{(}\FunctionTok{str\_replace\_all}\NormalTok{(}\StringTok{\textasciigrave{}}\AttributeTok{Operações de Crédito}\StringTok{\textasciigrave{}}\NormalTok{, }\StringTok{"}\SpecialCharTok{\textbackslash{}\textbackslash{}}\StringTok{."}\NormalTok{, }\StringTok{""}\NormalTok{))}\SpecialCharTok{/}\DecValTok{1000}\NormalTok{,}
  
   \AttributeTok{Subsetor =} \FunctionTok{case\_when}\NormalTok{(}
\NormalTok{              Tipo }\SpecialCharTok{==} \StringTok{"b1"} \SpecialCharTok{\textasciitilde{}} \StringTok{"Bancos comerciais"}\NormalTok{,}
\NormalTok{              Tipo }\SpecialCharTok{==} \StringTok{"b2"} \SpecialCharTok{\textasciitilde{}} \StringTok{"Bancos de investimento}\SpecialCharTok{\textbackslash{}n}\StringTok{ e câmbio"}\NormalTok{,}
\NormalTok{              Tipo }\SpecialCharTok{==} \StringTok{"b3C"} \SpecialCharTok{|} 
\NormalTok{              Tipo }\SpecialCharTok{==} \StringTok{"b3S"} \SpecialCharTok{\textasciitilde{}} \StringTok{"Cooperativas Singulares}\SpecialCharTok{\textbackslash{}n}\StringTok{ e Confederações"}\NormalTok{,}
\NormalTok{              Tipo }\SpecialCharTok{==} \StringTok{"b4"} \SpecialCharTok{\textasciitilde{}} \StringTok{"Bancos}\SpecialCharTok{\textbackslash{}n}\StringTok{ de desenvolvimento"}\NormalTok{,}
              \ConstantTok{TRUE} \SpecialCharTok{\textasciitilde{}} \StringTok{"Não bancários}\SpecialCharTok{\textbackslash{}n}\StringTok{ de crédito e capital"}
\NormalTok{                      ))}
\end{Highlighting}
\end{Shaded}

Ainda no pacote Dplyr, sumarizamos as informações conforme o script
abaixo e mostramos seu resultado em uma tabela do tipo Kable.

\begin{verbatim}
## `summarise()` has grouped output by 'Subsetor'. You can override using the
## `.groups` argument.
\end{verbatim}

Cada uma das variáveis criadas contêm os valores e infomações
analisados, neste caso, o total de crédito movimentado pelas
instituições e seu ativo acumulado. Podemos mostrar esses dados, como
mencionado, usando um kable, como se vê abaixo.

\begin{Shaded}
\begin{Highlighting}[]
\NormalTok{knitr}\SpecialCharTok{::}\FunctionTok{kable}\NormalTok{(}\FunctionTok{select}\NormalTok{(Dados.sum, }\SpecialCharTok{{-}}\DecValTok{2}\NormalTok{), }\AttributeTok{format=}\StringTok{"latex"}\NormalTok{, }\AttributeTok{booktabs =} \ConstantTok{TRUE}\NormalTok{) }\SpecialCharTok{\%\textgreater{}\%} 
  \FunctionTok{kable\_styling}\NormalTok{(}\AttributeTok{latex\_options=}\StringTok{"scale\_down"}\NormalTok{) }\SpecialCharTok{\%\textgreater{}\%}
  \FunctionTok{add\_header\_above}\NormalTok{(}\FunctionTok{c}\NormalTok{(}\StringTok{" "}\NormalTok{, }\StringTok{"Crédito"} \OtherTok{=} \DecValTok{3}\NormalTok{, }\StringTok{"Ativo"} \OtherTok{=} \DecValTok{3}\NormalTok{)) }
\end{Highlighting}
\end{Shaded}

\begin{table}
\centering
\resizebox{\linewidth}{!}{
\begin{tabular}{lrrrrrrr}
\toprule
\multicolumn{1}{c}{ } & \multicolumn{3}{c}{Crédito} & \multicolumn{3}{c}{Ativo} \\
\cmidrule(l{3pt}r{3pt}){2-4} \cmidrule(l{3pt}r{3pt}){5-7}
Subsetor & Número de instituições & Média de ativos & Desvio Padrão & Total dos ativos das instituições & Média de ativos  & Desvio Padrão  & Total das operações de crédito\\
\midrule
Bancos comerciais & 121 & 58930.5293 & 227646.2729 & 7130594.0 & 17154.48608 & 81475.1343 & 2075692.82\\
Bancos comerciais & 121 & 58444.0807 & 223699.4890 & 7071733.8 & 16857.67724 & 79820.3831 & 2039778.95\\
Bancos comerciais & 120 & 62584.5057 & 234742.8157 & 7510140.7 & 17480.89597 & 80042.5492 & 2097707.52\\
Bancos comerciais & 118 & 71688.6397 & 258908.6862 & 8459259.5 & 19228.93591 & 84359.0386 & 2269014.44\\
Bancos comerciais & 122 & 76856.2564 & 276577.0460 & 9376463.3 & 22371.07794 & 95243.7314 & 2729271.51\\
\addlinespace
Bancos de investimento
 e câmbio & 49 & 6099.7934 & 14953.8691 & 298889.9 & 2057.51961 & 3684.9182 & 100818.46\\
Bancos de investimento
 e câmbio & 47 & 6287.2402 & 14600.7006 & 295500.3 & 2259.36345 & 3901.2679 & 106190.08\\
Bancos de investimento
 e câmbio & 49 & 6409.4601 & 14316.2413 & 314063.5 & 2513.73073 & 4498.6350 & 123172.81\\
Bancos de investimento
 e câmbio & 50 & 7609.7054 & 16950.2313 & 380485.3 & 2974.40478 & 5382.8850 & 148720.24\\
Bancos de investimento
 e câmbio & 50 & 7501.1364 & 18408.2504 & 375056.8 & 2872.49282 & 5689.2304 & 143624.64\\
\addlinespace
Bancos
 de desenvolvimento & 4 & 222844.3417 & 428854.9195 & 891377.4 & 84648.26225 & 156240.4318 & 338593.05\\
Bancos
 de desenvolvimento & 4 & 220123.2657 & 423484.2391 & 880493.1 & 76407.34975 & 139933.8657 & 305629.40\\
Bancos
 de desenvolvimento & 4 & 218613.4378 & 420671.5729 & 874453.8 & 72797.95600 & 133418.3690 & 291191.82\\
Bancos
 de desenvolvimento & 4 & 191153.0200 & 366384.5756 & 764612.1 & 66662.08700 & 121633.8340 & 266648.35\\
Bancos
 de desenvolvimento & 4 & 200127.8682 & 382916.9837 & 800511.5 & 68933.38025 & 125228.9071 & 275733.52\\
\addlinespace
Cooperativas Singulares
 e Confederações & 1041 & 224.2726 & 653.5361 & 232570.7 & 76.49042 & 172.6136 & 79320.56\\
Cooperativas Singulares
 e Confederações & 1002 & 263.3706 & 722.4167 & 263634.0 & 92.57099 & 200.4928 & 92663.56\\
Cooperativas Singulares
 e Confederações & 946 & 318.4973 & 788.8557 & 300661.5 & 123.08610 & 257.8310 & 116193.28\\
Cooperativas Singulares
 e Confederações & 900 & 378.1611 & 828.6549 & 339588.7 & 162.43990 & 331.2519 & 145871.03\\
Cooperativas Singulares
 e Confederações & 875 & 525.4639 & 1103.6114 & 457153.6 & 233.88032 & 468.3193 & 203475.88\\
\addlinespace
Não bancários
 de crédito e capital & 384 & 1477.2030 & 10223.2066 & 561337.2 & 224.20178 & 2011.9586 & 85196.68\\
Não bancários
 de crédito e capital & 375 & 1494.4653 & 7523.9598 & 557435.5 & 253.56360 & 2341.0042 & 94579.22\\
Não bancários
 de crédito e capital & 375 & 1538.1929 & 7311.8353 & 570669.6 & 284.14846 & 2671.1375 & 105419.08\\
Não bancários
 de crédito e capital & 384 & 1547.3302 & 7122.8476 & 589532.8 & 310.38381 & 2945.1893 & 118256.23\\
Não bancários
 de crédito e capital & 414 & 1576.9031 & 7262.1632 & 635491.9 & 209.05560 & 2251.0614 & 84249.41\\
\bottomrule
\end{tabular}}
\end{table}

\hypertarget{metodologia}{%
\section{Metodologia}\label{metodologia}}

\hypertarget{relauxe7uxe3o-de-concentrauxe7uxe3o}{%
\subsection{Relação de
Concentração}\label{relauxe7uxe3o-de-concentrauxe7uxe3o}}

A Relação de Concentração (CR) mede a parcela de participação de um
número determinado de firmas no contexto geral da indústria examinada. A
fórmula abaixo ilustra a aplicação da técnica numa situação em que se
leva em conta as quatro maiores empresas do setor:

\[CR4 = \sum_{i = 4}^{k}{S_{i}} \]

Aqui, se aplicará a Relação de Concentração para as quatro e oito
maiores empresas de cada segmento. A tabela abaixo ajuda a compreender
os resultados.

\begin{longtable}[]{@{}lll@{}}
\toprule()
\textbf{Níveis de Mercado CR} & CR4 & CR8 \\
\midrule()
\endhead
Altamente Concentrado & i \textgreater{} 75\% & i \textgreater{} 90\% \\
Alta Concentração & 65\% \textless{} i \textless{} 75\% & 85\%
\textless{} i \textless{} 90\% \\
Concentração moderada & 50\% \textless{} i \textless{} 65\% & 70\%
\textless{} i \textless{} 85\% \\
Baixa Concentração & 35\% \textless{} i \textless{} 50\% & 45\%
\textless{} i \textless{} 70\% \\
Ausência de concentração & i \textless{} 35\% & i \textless{} 45\% \\
Claramente Atomístico & i = 2\% & \\
\bottomrule()
\end{longtable}

\hypertarget{uxedndice-de-hirshman-herfindahl-hhi}{%
\subsection{Índice de Hirshman-Herfindahl
(HHI)}\label{uxedndice-de-hirshman-herfindahl-hhi}}

Outra técnica para se medir concentração de mercado é o Índice de
HerfindahlHirschman (HHI), em que são somados os quadrados da parcela de
participação de cada empresa, levando-se em conta todas as empresas
conforme ilustra a expressão abaixo:

\[HHI = \sum_{i = 1}^{k}{S_{i}^2} \]

Ao contrário da relação CR, o Índice HHI leva em conta o tamanho
relativo das empresas ao elevar a parcela de participação de cada uma
delas ao quadrado. A tabela abaixo ajuda a compreendê-lo:

\begin{longtable}[]{@{}ll@{}}
\toprule()
Índice & Classificação do mercado \\
\midrule()
\endhead
HHI \textless{} 1000 & Não concentrado \\
HHI \textgreater{} 1000 porém \textless{} 1800 & Moderadamente
concentrado \\
HHI \textgreater{} 1800 & Altamente concentrado \\
\bottomrule()
\end{longtable}

No ambiente do R, vamos utilizar de um mecanismo provido pela
ferramentas em que fixamos um grupo (isto é, um eixo) e obtemos apenas
os maiores valores selecionados, nesta caso, as 4 e 8 maiores
instituições, respectivamente, que atuam em determinado segmento do
setor financeiro.

Segue-se, sequencialmente, os scripts que realizam esta tarefa tanto
para as operações de crédito quanto para os ativos detidos pelos bancos.
Os comandos abaixo têm, em essência, as mesmas funções, sendo estas as
responsáveis por determinada tarefa nos dados.

\begin{itemize}
\tightlist
\item
  \emph{group\_by()}, responsável por fixar os eixos ou grupos
  analisados;
\item
  \emph{slice\_max()}, obtem os maiores valores de cada grupo, recebendo
  como argumento a variável e quantidade de posições;
\item
  \emph{merge()}, mescla bases diferentes com características (ou
  variáveis) comuns, similar a uma chave primária;
\item
  \emph{mutate()}, cria novas variáveis a partir da base selecionada
  (cf. comando pipe);
\item
  \emph{summarise()}, agrega os valores conforme o grupo e operação
  selecionada, usamos sum() para somar os valores das maiores
  instituições.
\end{itemize}

1 - Obtendo a participação das 4 maiores empresas por setor, por
período, para o total de ativos financeiros.

\begin{Shaded}
\begin{Highlighting}[]
\NormalTok{CR4\_a }\OtherTok{\textless{}{-}}\NormalTok{ Dados.com }\SpecialCharTok{\%\textgreater{}\%} 
  \FunctionTok{group\_by}\NormalTok{(Subsetor, Período) }\SpecialCharTok{\%\textgreater{}\%} 
  \FunctionTok{slice\_max}\NormalTok{(}\StringTok{\textasciigrave{}}\AttributeTok{Ativo total}\StringTok{\textasciigrave{}}\NormalTok{, }\AttributeTok{n =} \DecValTok{4}\NormalTok{) }\SpecialCharTok{\%\textgreater{}\%}   
  
  \FunctionTok{merge}\NormalTok{(Dados.sum, }\AttributeTok{by =} \FunctionTok{c}\NormalTok{(}\StringTok{"Período"}\NormalTok{,}\StringTok{"Subsetor"}\NormalTok{), }\AttributeTok{all.x =}\NormalTok{ T) }\SpecialCharTok{\%\textgreater{}\%}
  \FunctionTok{mutate}\NormalTok{(Participação }\OtherTok{=}\NormalTok{ (}\StringTok{\textasciigrave{}}\AttributeTok{Ativo total}\StringTok{\textasciigrave{}}\SpecialCharTok{/}\StringTok{\textasciigrave{}}\AttributeTok{Total dos ativos das instituições}\StringTok{\textasciigrave{}}\NormalTok{)}\SpecialCharTok{*}\DecValTok{100}\NormalTok{) }\SpecialCharTok{\%\textgreater{}\%}
  \FunctionTok{group\_by}\NormalTok{(Subsetor, Período) }\SpecialCharTok{\%\textgreater{}\%}
  \FunctionTok{summarise}\NormalTok{(}\StringTok{\textasciigrave{}}\AttributeTok{RC Ativo}\StringTok{\textasciigrave{}} \OtherTok{=} \FunctionTok{sum}\NormalTok{(Participação))}
\end{Highlighting}
\end{Shaded}

\begin{verbatim}
## `summarise()` has grouped output by 'Subsetor'. You can override using the
## `.groups` argument.
\end{verbatim}

2 - Obtendo a participação das 4 maiores empresas por operações de
créditos realizadas no período.

\begin{Shaded}
\begin{Highlighting}[]
\NormalTok{CR4\_c }\OtherTok{\textless{}{-}}\NormalTok{ Dados.com }\SpecialCharTok{\%\textgreater{}\%} 
  \FunctionTok{group\_by}\NormalTok{(Subsetor, Período) }\SpecialCharTok{\%\textgreater{}\%} 
  \FunctionTok{slice\_max}\NormalTok{(}\StringTok{\textasciigrave{}}\AttributeTok{Operações de Crédito}\StringTok{\textasciigrave{}}\NormalTok{, }\AttributeTok{n =} \DecValTok{4}\NormalTok{) }\SpecialCharTok{\%\textgreater{}\%}   
  
  \FunctionTok{merge}\NormalTok{(Dados.sum, }\AttributeTok{by =} \FunctionTok{c}\NormalTok{(}\StringTok{"Período"}\NormalTok{,}\StringTok{"Subsetor"}\NormalTok{), }\AttributeTok{all.x =}\NormalTok{ T) }\SpecialCharTok{\%\textgreater{}\%}
  \FunctionTok{mutate}\NormalTok{(Participação }\OtherTok{=}\NormalTok{ (}\StringTok{\textasciigrave{}}\AttributeTok{Operações de Crédito}\StringTok{\textasciigrave{}}\SpecialCharTok{/}
                         \StringTok{\textasciigrave{}}\AttributeTok{Total das operações de crédito}\StringTok{\textasciigrave{}}\NormalTok{)}\SpecialCharTok{*}\DecValTok{100}\NormalTok{) }\SpecialCharTok{\%\textgreater{}\%}
  \FunctionTok{group\_by}\NormalTok{(Subsetor, Período) }\SpecialCharTok{\%\textgreater{}\%}
  \FunctionTok{summarise}\NormalTok{(}\StringTok{\textasciigrave{}}\AttributeTok{RC Crédito}\StringTok{\textasciigrave{}} \OtherTok{=} \FunctionTok{sum}\NormalTok{(Participação)) }\SpecialCharTok{\%\textgreater{}\%} \FunctionTok{mutate}\NormalTok{(}\AttributeTok{Indicador =} \StringTok{"CR4"}\NormalTok{)}
\end{Highlighting}
\end{Shaded}

\begin{verbatim}
## `summarise()` has grouped output by 'Subsetor'. You can override using the
## `.groups` argument.
\end{verbatim}

\begin{Shaded}
\begin{Highlighting}[]
\CommentTok{\# juntando as duas bases}
\NormalTok{CR4 }\OtherTok{\textless{}{-}} \FunctionTok{merge}\NormalTok{(CR4\_a, CR4\_c, }\AttributeTok{by =} \FunctionTok{c}\NormalTok{(}\StringTok{"Período"}\NormalTok{,}\StringTok{"Subsetor"}\NormalTok{))}
\end{Highlighting}
\end{Shaded}

3 - Obtendo a participação das 8 maiores empresas por setor (CR8), por
período, para o total de ativos financeiros.

\begin{Shaded}
\begin{Highlighting}[]
\NormalTok{CR8\_a }\OtherTok{\textless{}{-}}\NormalTok{ Dados.com }\SpecialCharTok{\%\textgreater{}\%} 
  \FunctionTok{group\_by}\NormalTok{(Subsetor, Período) }\SpecialCharTok{\%\textgreater{}\%} 
  \FunctionTok{slice\_max}\NormalTok{(}\StringTok{\textasciigrave{}}\AttributeTok{Ativo total}\StringTok{\textasciigrave{}}\NormalTok{, }\AttributeTok{n =} \DecValTok{8}\NormalTok{) }\SpecialCharTok{\%\textgreater{}\%}
  
  \FunctionTok{merge}\NormalTok{(Dados.sum, }\AttributeTok{by =} \FunctionTok{c}\NormalTok{(}\StringTok{"Período"}\NormalTok{,}\StringTok{"Subsetor"}\NormalTok{), }\AttributeTok{all.x =}\NormalTok{ T) }\SpecialCharTok{\%\textgreater{}\%}
  \FunctionTok{mutate}\NormalTok{(Participação }\OtherTok{=}\NormalTok{ (}\StringTok{\textasciigrave{}}\AttributeTok{Ativo total}\StringTok{\textasciigrave{}}\SpecialCharTok{/}
                         \StringTok{\textasciigrave{}}\AttributeTok{Total dos ativos das instituições}\StringTok{\textasciigrave{}}\NormalTok{)}\SpecialCharTok{*}\DecValTok{100}\NormalTok{) }\SpecialCharTok{\%\textgreater{}\%}
  \FunctionTok{group\_by}\NormalTok{(Subsetor, Período) }\SpecialCharTok{\%\textgreater{}\%}
  \FunctionTok{summarise}\NormalTok{(}\StringTok{\textasciigrave{}}\AttributeTok{RC Ativo}\StringTok{\textasciigrave{}} \OtherTok{=} \FunctionTok{sum}\NormalTok{(Participação))}
\end{Highlighting}
\end{Shaded}

\begin{verbatim}
## `summarise()` has grouped output by 'Subsetor'. You can override using the
## `.groups` argument.
\end{verbatim}

4 - Obtendo a participação das 4 maiores empresas por operações de
créditos realizadas no período.

\begin{Shaded}
\begin{Highlighting}[]
\NormalTok{CR8\_c }\OtherTok{\textless{}{-}}\NormalTok{ Dados.com }\SpecialCharTok{\%\textgreater{}\%} 
  \FunctionTok{group\_by}\NormalTok{(Subsetor, Período) }\SpecialCharTok{\%\textgreater{}\%} 
  \FunctionTok{slice\_max}\NormalTok{(}\StringTok{\textasciigrave{}}\AttributeTok{Operações de Crédito}\StringTok{\textasciigrave{}}\NormalTok{, }\AttributeTok{n =} \DecValTok{8}\NormalTok{) }\SpecialCharTok{\%\textgreater{}\%}
  
  \FunctionTok{merge}\NormalTok{(Dados.sum, }\AttributeTok{by =} \FunctionTok{c}\NormalTok{(}\StringTok{"Período"}\NormalTok{,}\StringTok{"Subsetor"}\NormalTok{), }\AttributeTok{all.x =}\NormalTok{ T) }\SpecialCharTok{\%\textgreater{}\%}
  \FunctionTok{mutate}\NormalTok{(Participação }\OtherTok{=}\NormalTok{ (}\StringTok{\textasciigrave{}}\AttributeTok{Operações de Crédito}\StringTok{\textasciigrave{}}\SpecialCharTok{/}
                         \StringTok{\textasciigrave{}}\AttributeTok{Total das operações de crédito}\StringTok{\textasciigrave{}}\NormalTok{)}\SpecialCharTok{*}\DecValTok{100}\NormalTok{) }\SpecialCharTok{\%\textgreater{}\%}
  \FunctionTok{group\_by}\NormalTok{(Subsetor, Período) }\SpecialCharTok{\%\textgreater{}\%}
  \FunctionTok{summarise}\NormalTok{(}\StringTok{\textasciigrave{}}\AttributeTok{RC Crédito}\StringTok{\textasciigrave{}} \OtherTok{=} \FunctionTok{sum}\NormalTok{(Participação)) }\SpecialCharTok{\%\textgreater{}\%} \FunctionTok{mutate}\NormalTok{(}\AttributeTok{Indicador =} \StringTok{"CR8"}\NormalTok{)}
\end{Highlighting}
\end{Shaded}

\begin{verbatim}
## `summarise()` has grouped output by 'Subsetor'. You can override using the
## `.groups` argument.
\end{verbatim}

\begin{Shaded}
\begin{Highlighting}[]
\NormalTok{CR8 }\OtherTok{\textless{}{-}} \FunctionTok{merge}\NormalTok{(CR8\_a, CR8\_c, }\AttributeTok{by =} \FunctionTok{c}\NormalTok{(}\StringTok{"Período"}\NormalTok{,}\StringTok{"Subsetor"}\NormalTok{))}
\end{Highlighting}
\end{Shaded}

5 - Consolidando todas as informações em um só data.frame.

\begin{Shaded}
\begin{Highlighting}[]
\NormalTok{OC }\OtherTok{\textless{}{-}} \FunctionTok{bind\_rows}\NormalTok{(CR4, CR8)}

\NormalTok{OC }\SpecialCharTok{\%\textless{}\textgreater{}\%} \FunctionTok{pivot\_longer}\NormalTok{(}\AttributeTok{cols =} \StringTok{\textasciigrave{}}\AttributeTok{RC Ativo}\StringTok{\textasciigrave{}}\SpecialCharTok{:}\StringTok{\textasciigrave{}}\AttributeTok{RC Crédito}\StringTok{\textasciigrave{}}\NormalTok{, }
                     \AttributeTok{values\_to =} \StringTok{"Participação"}\NormalTok{, }
                     \AttributeTok{names\_to =} \StringTok{"CR"}\NormalTok{)}

\CommentTok{\# retirando excessos}
\NormalTok{gdata}\SpecialCharTok{::}\FunctionTok{keep}\NormalTok{(Dados.com, OC, Dados.sum, }\AttributeTok{sure =}\NormalTok{ T)}
\end{Highlighting}
\end{Shaded}

Pronto! Consolidamos a base correspondente a relação de concentração.
Agora é tarefa nossa realizar as operações para obter o Índice
Hirschman-Herfindahl.

Como mencionado antes, este índice toma como base a participação das
empresas e as eleva ao quadrado, somando todos os valores da base. Para
os dados que dispomos, esta operação é similar àquela realizada para a
relação de concentração, com a diferença da operação matemática e do uso
de todas as instituições dos segmentos, não apenas os maiores.

Desta forma, realizamos isso em três etapas simples:

1 - Sumarizando os valores para cada empresa por segmento (participação
individual)\footnote{Perceba que aqui nós já aplicamos a exponenciação
  da participação}.

\begin{Shaded}
\begin{Highlighting}[]
\NormalTok{HHI }\OtherTok{\textless{}{-}}\NormalTok{ Dados.com }\SpecialCharTok{\%\textgreater{}\%}
  \FunctionTok{group\_by}\NormalTok{(Período, Subsetor) }\SpecialCharTok{\%\textgreater{}\%}
  \FunctionTok{summarise}\NormalTok{(HHI\_crédito }\OtherTok{=}\NormalTok{ (}\StringTok{\textasciigrave{}}\AttributeTok{Operações de Crédito}\StringTok{\textasciigrave{}}\SpecialCharTok{/}
                            \FunctionTok{sum}\NormalTok{(}\StringTok{\textasciigrave{}}\AttributeTok{Operações de Crédito}\StringTok{\textasciigrave{}}\NormalTok{, }\AttributeTok{na.rm =}\NormalTok{ T)}\SpecialCharTok{*}\DecValTok{100}\NormalTok{)}\SpecialCharTok{\^{}}\DecValTok{2}\NormalTok{,}
            \AttributeTok{HHI\_ativo =}\NormalTok{ (}\StringTok{\textasciigrave{}}\AttributeTok{Ativo total}\StringTok{\textasciigrave{}}\SpecialCharTok{/}
                         \FunctionTok{sum}\NormalTok{(}\StringTok{\textasciigrave{}}\AttributeTok{Ativo total}\StringTok{\textasciigrave{}}\NormalTok{, }\AttributeTok{na.rm =}\NormalTok{ T)}\SpecialCharTok{*}\DecValTok{100}\NormalTok{)}\SpecialCharTok{\^{}}\DecValTok{2}\NormalTok{)}
\end{Highlighting}
\end{Shaded}

\begin{verbatim}
## `summarise()` has grouped output by 'Período', 'Subsetor'. You can override
## using the `.groups` argument.
\end{verbatim}

2 - Agregando os valores por segmento, obtendo o HHI pela função
\emph{sum()} (somatório).

\begin{Shaded}
\begin{Highlighting}[]
\NormalTok{HHI }\SpecialCharTok{\%\textless{}\textgreater{}\%}
  \FunctionTok{group\_by}\NormalTok{(Período, Subsetor) }\SpecialCharTok{\%\textgreater{}\%}
  \FunctionTok{summarise}\NormalTok{(Crédito }\OtherTok{=} \FunctionTok{sum}\NormalTok{(HHI\_crédito, }\AttributeTok{na.rm =}\NormalTok{ T),}
            \StringTok{\textasciigrave{}}\AttributeTok{Ativo}\StringTok{\textasciigrave{}} \OtherTok{=} \FunctionTok{sum}\NormalTok{(HHI\_ativo, }\AttributeTok{na.rm =}\NormalTok{ T))}
\end{Highlighting}
\end{Shaded}

\begin{verbatim}
## `summarise()` has grouped output by 'Período'. You can override using the
## `.groups` argument.
\end{verbatim}

3 - Por último, nós juntamos as variáveis para facilitar a visualização
no gráfico.

\hypertarget{resultados}{%
\section{Resultados}\label{resultados}}

\begin{Shaded}
\begin{Highlighting}[]
\NormalTok{p }\OtherTok{\textless{}{-}} 
  \FunctionTok{filter}\NormalTok{(OC, Subsetor }\SpecialCharTok{!=} \StringTok{"Bancos}\SpecialCharTok{\textbackslash{}n}\StringTok{ de desenvolvimento"}\NormalTok{) }\SpecialCharTok{\%\textgreater{}\%}
  \FunctionTok{ggplot}\NormalTok{(}\FunctionTok{aes}\NormalTok{(}\AttributeTok{x =}\NormalTok{ Período, }\AttributeTok{y =}\NormalTok{ Participação)) }\SpecialCharTok{+}
  \FunctionTok{geom\_bar}\NormalTok{(}
    \FunctionTok{aes}\NormalTok{(}\AttributeTok{color =}\NormalTok{ Indicador, }\AttributeTok{fill =}\NormalTok{ Indicador),}
    \AttributeTok{stat =} \StringTok{"identity"}\NormalTok{, }\AttributeTok{position =} \FunctionTok{position\_dodge}\NormalTok{(}\FloatTok{0.8}\NormalTok{),}
    \AttributeTok{width =} \FloatTok{0.7}
\NormalTok{    ) }\SpecialCharTok{+}
  \FunctionTok{theme}\NormalTok{(}\AttributeTok{axis.text.x =} \FunctionTok{element\_text}\NormalTok{(}\AttributeTok{size =} \DecValTok{10}\NormalTok{, }
                                   \AttributeTok{color =} \StringTok{\textquotesingle{}black\textquotesingle{}}\NormalTok{, }
                                   \AttributeTok{angle =} \DecValTok{90}\NormalTok{, }
                                   \AttributeTok{vjust =}\NormalTok{ .}\DecValTok{3}\NormalTok{))}

\NormalTok{p }\OtherTok{\textless{}{-}}\NormalTok{ p }\SpecialCharTok{+} \FunctionTok{geom\_text}\NormalTok{(}
  \FunctionTok{aes}\NormalTok{(}\AttributeTok{label =} \FunctionTok{round}\NormalTok{(Participação,}\DecValTok{0}\NormalTok{), }\AttributeTok{group =}\NormalTok{ Indicador), }
  \AttributeTok{position =} \FunctionTok{position\_dodge}\NormalTok{(}\FloatTok{0.8}\NormalTok{),}
  \AttributeTok{vjust =} \SpecialCharTok{{-}}\FloatTok{0.3}\NormalTok{, }\AttributeTok{size =} \FloatTok{3.5}
\NormalTok{) }

\NormalTok{p }\SpecialCharTok{+} \FunctionTok{facet\_grid}\NormalTok{(CR}\SpecialCharTok{\textasciitilde{}}\NormalTok{Subsetor) }\SpecialCharTok{+} \FunctionTok{theme\_bw}\NormalTok{() }
\end{Highlighting}
\end{Shaded}

\includegraphics{Script-CB---df_files/figure-latex/unnamed-chunk-16-1.pdf}

\begin{Shaded}
\begin{Highlighting}[]
\NormalTok{test }\OtherTok{\textless{}{-}}
\FunctionTok{ggplot}\NormalTok{(HHI, }\AttributeTok{stat =} \StringTok{\textquotesingle{}identity\textquotesingle{}}\NormalTok{, }\AttributeTok{position =} \StringTok{"dodge"}\NormalTok{, }
  \FunctionTok{aes}\NormalTok{(}\AttributeTok{x =}\NormalTok{ Período, }\AttributeTok{y =}\NormalTok{ Índice, }\AttributeTok{group =}\NormalTok{ Tipo, }\AttributeTok{label =} \FunctionTok{round}\NormalTok{(Índice,}\DecValTok{0}\NormalTok{), }\AttributeTok{fill =}\NormalTok{ Tipo)) }\SpecialCharTok{+}
  \FunctionTok{geom\_line}\NormalTok{(}\AttributeTok{alpha =} \FloatTok{0.8}\NormalTok{, }\AttributeTok{stroke =} \DecValTok{0}\NormalTok{, }\AttributeTok{size =} \FloatTok{1.1}\NormalTok{) }\SpecialCharTok{+}
  \FunctionTok{geom\_point}\NormalTok{() }\SpecialCharTok{+}
  \FunctionTok{geom\_label}\NormalTok{(}\AttributeTok{segment.colour=}\StringTok{"black"}\NormalTok{, }\AttributeTok{size =}\FloatTok{3.1}\NormalTok{) }\SpecialCharTok{+} 
  \FunctionTok{facet\_wrap}\NormalTok{(}\SpecialCharTok{\textasciitilde{}}\NormalTok{Subsetor, }\AttributeTok{scales =} \StringTok{"free\_y"}\NormalTok{) }\SpecialCharTok{+}
\FunctionTok{labs}\NormalTok{(}\AttributeTok{title =} \StringTok{"Concentração Bancária {-} Índice de Hirschman{-}Herfindahl (HHI)"}\NormalTok{, }
     \AttributeTok{subtitle =} \StringTok{"1º tri de 2017{-}2021"}\NormalTok{,}
     \AttributeTok{x =} \StringTok{"Período"}\NormalTok{,}
     \AttributeTok{y =} \StringTok{"Pontos HHI"}\NormalTok{) }\SpecialCharTok{+}
  \FunctionTok{theme\_bw}\NormalTok{() }\SpecialCharTok{+}
        \FunctionTok{theme}\NormalTok{(}\AttributeTok{axis.title =} \FunctionTok{element\_text}\NormalTok{(}\AttributeTok{size =} \DecValTok{12}\NormalTok{),}
        \AttributeTok{axis.title.y =} \FunctionTok{element\_text}\NormalTok{(}\AttributeTok{size =} \FloatTok{11.5}\NormalTok{),                 }
        \AttributeTok{axis.text.x =} \FunctionTok{element\_text}\NormalTok{(}\AttributeTok{size =} \DecValTok{10}\NormalTok{, }\AttributeTok{color =} \StringTok{\textquotesingle{}black\textquotesingle{}}\NormalTok{, }\AttributeTok{angle =} \DecValTok{90}\NormalTok{, }\AttributeTok{vjust =}\NormalTok{ .}\DecValTok{3}\NormalTok{),}
        \AttributeTok{axis.text.y =} \FunctionTok{element\_text}\NormalTok{(}\AttributeTok{size =} \DecValTok{11}\NormalTok{, }\AttributeTok{color =} \StringTok{\textquotesingle{}black\textquotesingle{}}\NormalTok{, }\AttributeTok{face =} \StringTok{\textquotesingle{}bold\textquotesingle{}}\NormalTok{),}
        \AttributeTok{legend.title =} \FunctionTok{element\_text}\NormalTok{(}\AttributeTok{size =} \DecValTok{12}\NormalTok{),}
        \AttributeTok{legend.position=}\StringTok{\textquotesingle{}bottom\textquotesingle{}}\NormalTok{,}
        \AttributeTok{legend.text=}\FunctionTok{element\_text}\NormalTok{(}\AttributeTok{size=}\DecValTok{13}\NormalTok{),}
        \AttributeTok{legend.background =} \FunctionTok{element\_rect}\NormalTok{(}\AttributeTok{fill=}\StringTok{"ghostwhite"}\NormalTok{, }\AttributeTok{size=}\DecValTok{1}\NormalTok{),}
        \AttributeTok{axis.ticks =} \FunctionTok{element\_line}\NormalTok{(}\AttributeTok{colour =} \StringTok{"grey70"}\NormalTok{, }\AttributeTok{size =} \FloatTok{0.2}\NormalTok{),}
        \AttributeTok{panel.grid.major =} \FunctionTok{element\_line}\NormalTok{(}\AttributeTok{colour =} \StringTok{"grey70"}\NormalTok{),}
        \AttributeTok{strip.text =} \FunctionTok{element\_text}\NormalTok{(}\AttributeTok{face =} \StringTok{\textquotesingle{}bold\textquotesingle{}}\NormalTok{)) }
  
\NormalTok{test}
\end{Highlighting}
\end{Shaded}

\includegraphics{Script-CB---df_files/figure-latex/unnamed-chunk-17-1.pdf}

\end{document}
